\documentclass{article}
\usepackage{graphicx}

\begin{document}

\title{Chapter 2}
\author{Manish Shankla}

\maketitle


\section{Chapter 2}

\subsection{2.1}
Exercise 2.1 In $\epsilon$-greedy action selection, for the case of two actions and  $\epsilon$-= 0.5, what is the probability that the greedy action is selected?

Answer: 
\begin{equation}
P(a = a_1) = 0.5
\end{equation}

\subsection{2.2}

Exercise 2.2: 

Answer: Time-step 2 and 5 the random action was chosen and time-step 3 it may have been chosen state since the average reward for state 2 was 0.5 at time-step 3. 

\subsection{2.3}

Exercise 2.3: 

Answer: $\epsilon = 0.1$. Randomness promotes exploration and 0.1 is still a small probability of picking a random state. 



\subsection{2.4}
Exercise 2.4 If the step-size parameters, ↵n, are not constant, then the estimate Qn is a weighted average of previously received rewards with a weighting di↵erent from that given by (2.6). What is the weighting on each prior reward for the general case, analogous to (2.6), in terms of the sequence of step-size parameters?




\end{document}
